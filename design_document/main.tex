\documentclass[12pt, letterpaper]{article}
\usepackage{graphicx}
\usepackage[table]{xcolor}
\usepackage{array,booktabs,ragged2e}
\usepackage{hyperref}
\usepackage{enumitem}
\graphicspath{{images/}}

\title{\textbf{Game Design Document}}
\author{Elicatza}
\date{July 2024}

\newcolumntype{R}[1]{>{\RaggedLeft\arraybackslash}p{#1}}
\setcounter{secnumdepth}{0}

\begin{document}
\maketitle
\tableofcontents
\section{Introduction}

My submission for the 
\href{https://itch.io/jam/pirate}{Pirate Software game jam 2024}.

\subsection{Summary}
Cube managing internal alchemy in a puzzling and shadowy world.

\subsection{Theme interpretation}

The themes are \textit{Alchemy and Shadows}. I'm choosing to interpret alchemy
in the direction of transformations. Specifically gender transformations and
changes in internal alchemy. Aesthetically will display a more literal
interpretation of alchemy.

Possibly the Taoist use of alchemy to attain immortality with focus on
spirituality.

In regards to shadows, forceful isolation --- an undesired retreat from social
life --- will be my interpretation. Contrast between light and dark will
be employed with shadows to emphasise this effect.

My hopes with shadows is to be able to showcase the turning point
when isolation no longer remains sufficient. Like Zarathustra, there
comes a time for sharing your blossoming soul.

\subsection{Inspiration}

Nietzsche, Laozi (Tao Te Ching), Spinoza.

\subsection{Software}
\begin{itemize}[itemsep=1pt]
    \item Raylib C
    \item Aseprite
    \item NeoVim
\end{itemize}

\subsection{Target Audience}

Myself

\section{Concept}

Once the theme is released:
\begin{itemize}[itemsep=1pt]
    \item Figure out which story to tell.
    \item Create gameplay to aid story.
    \item Use Aseprite to find artstyle.
    \item I've never done audio before.
\end{itemize}

\subsection{Gameplay}

Life is a puzzle. The gameplay should reflect that.
From completing them, you gain an understanding of the world and
therefore yourself.

The player will be presented with several puzzles / actions. When completing
tasks, the player learns how the character responds. Mostly bad. A pain and
energy meter is what governs what actions are possible. You'll have to balance
pain, energy and socialization. As a game maker I want the player to seek out
socialization, which will progress the game, but causes fatigue and pain.
Shadow and fatigue are equal. Pain represents all other ways of being hurt.

Puzzles that improve health and energy, will be boring, but necessary to
achieve development. Downtime like this should invoke the player to think. The
actions that cause development hurts you. Very \textit{Tao Te Ching} if you
will. Solving the mystery. Dying offers no way to replay (delete cookies to
restart).

\subsection{Mechanics}

You are a cube. Glide around with arrow keys (wasd). Roll around with shift +
arrow keys (wasd). Rolling causes pain. The ground will have different heights,
forcing the player to roll. 

During activities (puzzles), you'll have the additional ability to split the
cube and causing pain. The other option is to mirror shape across line.

There will be two (or more) gameplay modes utilizing these mechanics. In one
you create shapes, in another you traverse area, getting shapes to holes. Splits
that do not end in holes, hurt you.

\subsection{Story}

TODO

\section{Art}

TODO


\section{Audio}

TODO

Don't have a microphone, nor software for creating music.
I've heard silent films are all the rage, but I might have to look into
\href{https://blog.archive.org/2022/01/01/welcoming-recorded-music-to-the-public-domain/}{public domain music}.

\section{Roadmap}

\bgroup
\def\arraystretch{1.5}
\begin{tabular}{ R{0.4cm} p{4cm} p{1.5cm} p{5.5cm} }
\toprule
\multicolumn{1}{l}{\textbf{\#}} & \textbf{Assignment} & \textbf{Date} & \textbf{Notes} \\ 
\midrule
1 & Write roadmap & 07-17 & Basic overview of steps \\
2 & Write GDD & 07-17 & Figure out themes and content \\
3 & Basic build & 07-18 & C + Emscripten + Raylib fun \\
4 & Quick puzzle demo & TODO & Figure out what works \\
5 & Finalize gameplay & TODO & Make adjustments to idea \\
6 & Puzzle file formats & TODO & This will be embedded \\
7 & Design environment & TODO & Where the player moves \\
8 & Environment format & TODO & Embedded file format \\

X & Story & TODO & Write story \\
X & Balancing & TODO & Give user ``good'' incentives \\
X & UI & TODO & Start, pause, end \& status \\
X & ART style & TODO & How should the game look? \\
X & title & TODO & something something \\
X & Submit & TODO & And we're off \\

\bottomrule
\end{tabular}
\egroup

\end{document}
